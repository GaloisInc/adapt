

\documentclass{article}

\usepackage{hyperref}

\usepackage{fullpage}
\usepackage{a4}

\begin{document}
\title{A proposal for a baseline segmenter}
\maketitle

We plan to implement an extremely simple segmenter with the aim
of having a starting point for experimentation. This will hopefully
help us interact with the other project groups to identify requirements
for the segmenter and, in general, 
concretize our research plan.

Our segmenter extracts several subgraps from a given 
RDF graph according to a time span.
An RDF graph is simply a labeled directed graph, 
i.e. a set of edges of the form $s \to^p o$,
and hence RDF graphs are tipically represented 
as a set of triples of the form $(s,p,o)$.
\\[2ex]
More concretely, our segmenter takes as input
\begin{enumerate}
\item an RDF graph $G$ in Turtle format and
\item an interval duration $T$ (in microseconds),
\end{enumerate}
and produces the sequence of RDF graphs that corresponds to {\em the time-based segmentation of $G$ according to $T$}.

Let us now describe what we mean by {\em the time-based segmentation of $G$ according to $T$}.
As mentioned above, $G$ is a set of ordered triples $(s, p, o)$,
where $s$, $p$, and $o$ intuitively stand for subject, predicate, and object, respectively.
For example, the following are triples in turtle format
generated by the PARC team using the approach described in 
Manolis Stamatogiannakis et al.'s paper ``Decoupling Provenance Capture and
Analysis from Execution''~\cite{stamatogiannakis15tapp}.

{\scriptsize
\begin{verbatim}
<exe://bash~2819~2723> prov:startedAtTime 1519337 .
<exe://bash~2819~2723> prov:endedAtTime 1820209 . 
<file:/usr/lib/locale/locale-archive> adapt:wasAccessedBy <exe://mkdir~2820~2819> . 
<file:/tmp/prov\_files/alpaca.txt> prov:wasGeneratedBy <exe://bash~2824~2819> .
\end{verbatim}
}

Our segmenter exploits the information provided 
by the predicates startedAtTime and endedAtTime
to produce a list of subgraphs of $G$.
The RDF graph generated by the PARC team referred above
contains the relations: 
type, startedAtTime,
wasAccessedBy,
endedAtTime,
wasGeneratedBy,
wasDerivedFrom,
label, and
used.
Note that the only triples that refer to time 
are the ones with predicates
startedAtTime and endedAtTime.
Also, let us remark that not every 
subject in the graph has a 
startedAtTime and endedAtTime
predicate.
\\[2ex]
Our segmenter proceeds as follows:

1) The minimum $A$ and maximum $B$ absolute times mentioned in $G$ are computed.

2) Let $L = (a_1, b_1), \ldots, (a_n, b_n)$ be the list of all consecutive non-overlapping
time intervals of duration $T$ between $A$ and $B$. For each $(a_i, b_i)$ the segmenter ouputs an RDF graph $G_i$ constructed as follows:

(a) If there is a triple $(s, startedAtTime, o)\in G$ such that $o>b_i$, the subject $s$ is marked as {\em unstarted}.

(b) If there is a triple $(s, endedAtTime, o)\in G$ such that $o>a_i$, $s$ is marked as {\em unfinished}.

(c) Subjects, objects, and triples are marked according to the following rules:
\begin{itemize}
\item[R1.] If there is a triple $(s, wasGeneratedBy, o)\in G$ 
such that $o$ is unstarted then $s$ is marked as unstarted.
\item[R2.] If there is a triple $(s, wasDerivedFrom, o)\in G$ 
such that $o$ is unstarted then $s$ is marked as unstarted.
\item[R3.] A triple $(s, used, o)$ such that $o$ is unstarted is marked as unstarted.
\item[R4.] A triple $(s, wasAccessedBy, o)$ such that $o$ is unstarted is marked as unstarted.
\end{itemize}
(d) $G_i$ is obtained from $G$ by, after having applied the rules R1-R4 exhaustively, 
removing unstarted triples and triples with unstarted subject or object.

Similar rules can be added for subjects marked as {\em finished}, one could also think of 
other markings, such as {\em ongoing} or {\em expired}, with their corresponding rules.

\bibliographystyle{plain}
\bibliography{bib}

\end{document}
